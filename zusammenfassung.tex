\documentclass[11pt,a4paper]{scrartcl}

\usepackage[utf8]{inputenc}
\usepackage[T1]{fontenc}
\usepackage[ngerman]{babel}
\usepackage{amsmath,amsthm,amssymb,dsfont}
\usepackage{mathtools}
\usepackage[paper=a4paper,left=25mm,right=25mm,top=25mm,bottom=25mm]{geometry}
\usepackage{float}
\usepackage{hyperref}
\usepackage{enumerate}
\usepackage{url}
\usepackage{tikz}
\usepackage{esint}
\usepackage{csquotes}
\usepackage{textcomp}

\usepackage{setspace}

\parindent 0pt
\linespread{1.5}

% Makros

\newcommand{\N}{\mathbb{N}} % natuerliche Zahlen
\newcommand{\Z}{\mathbb{Z}} % ganze Zahlen
\newcommand{\Q}{\mathbb{Q}} % rationale Zahlen
\newcommand{\R}{\mathbb{R}} % reelle Zahlen
\newcommand{\K}{\mathbb{K}} % Körper
\newcommand{\C}{\mathbb{C}} % komplexe Zahlen
\newcommand{\D}{\mathcal{D}}
\newcommand{\E}{\mathcal{E}}
\newcommand{\Hc}{\mathcal{H}}
\newcommand{\Sc}{\mathcal{S}}
\newcommand{\A}{\mathcal{A}}
\newcommand{\B}{\mathcal{B}}
\newcommand{\M}{\mathcal{M}}
\newcommand{\Nc}{\mathcal{N}}
\newcommand{\F}{\mathcal{F}}
\newcommand{\norm}[1]{\|#1\|}
\newcommand{\laplace}{\triangle}
\newcommand{\circum}{\text{\textasciicircum}}

% Umgebungen für Definitionen, Sätze, usw.

\theoremstyle{plain}
\newtheorem{thm}{Satz}[section]
\newtheorem*{lem}{Lemma}
\newtheorem{cor}[thm]{Korollar}
\newtheorem{prop}[thm]{Proposition}
\newtheorem*{ex}{Beispiel}
\newtheorem*{ntion}{Notation}

\theoremstyle{definition}
\newtheorem{defn}[thm]{Definition}

\theoremstyle{remark}
\newtheorem*{bem}{Bemerkung}
\newtheorem{bemnumber}[thm]{Bemerkung}

\def\Satzrefname{Satz}

\DeclareMathOperator{\supp}{supp}
\DeclareMathOperator{\esssupp}{ess supp}
\DeclareMathOperator{\id}{id}
\DeclareMathOperator{\loc}{loc}
\DeclareMathOperator{\pv}{pv}
\DeclareMathOperator{\grad}{grad}
\DeclareMathOperator{\Bil}{Bil}
\DeclareMathOperator{\Lin}{Lin}
\DeclareMathOperator{\Mult}{Mult}
\DeclareMathOperator{\LH}{LH}
\DeclareMathOperator{\Rang}{Rang}
\DeclareMathOperator{\Spur}{Spur}
\DeclareMathOperator{\vN}{vN}

\begin{document}

\title{Zusammenfassung von Neumann Algebren}
\author{Sebastian Bechtel}
\maketitle

\section{Erster Kontakt}

Eine *-Algebra $\M \subset \B(\Hc)$ heißt \emph{Operatoralgebra}. Ist $\M$ stop-abgeschlossen, so heißt $\M$ \emph{von Neumann Algebra (vNA)}. Durch GNS-Darstellung lässt sich jede C*-Algebra als Operatoralgebra darstellen. Sei $\Sc \subset \B(\Hc)$, da der Schnitt von vNAen wieder vNA ist, existiert kleinste vNA $\vN(\Sc)$, die $\Sc$ enthält. Ist $\Sc$ Operatoralgebra, so gilt $\vN(\Sc)=\overline{\Sc}^{stop}$, dies ist aber nicht offensichtlich, da Involution nicht stop-stetig ist!

Für $\M \subset \B(\Hc)$ ist die \emph{Kommutante} gegeben durch $\M'\coloneqq \{ x\in\B(\Hc): xy=yx \text{ für all } y\in \M \}$. Es gilt immer $\M'$ stop-abgeschlossen und ist $\M$ Operatoralgebra, so auch $\M'$, somit $\M'$ vNA. Ferner gilt $1_\Hc\in \M'$ und $\M'=\M'''$ (wegen $\M\subset \M''$), also gilt für $\M$ Operatoralgebra auch nach Bikommutantensatz (vgl. später) $\M'$ vNA! Aus $\M \subset \M''$ folgt für $\M$ Operatoralgebra, dass $\vN(\M)\subset \M''$. Ist $1_\Hc\in M$, so folgt mit Bikommutantensatz $\M''\subset \vN(\M)''=\vN(\M)$, also $\M''=\vN(\M)$. Ist $\M$ nicht s.a., geht alles schief: Die Matrizen $\left(\begin{smallmatrix} 0 & a \\ 0 & 0 \end{smallmatrix}\right)$ haben als Kommutante $\left(\begin{smallmatrix} b & c \\ 0 & b \end{smallmatrix}\right)$, also ist jene nicht s.a., somit folgt wegen $\M'=\M'''$, dass Bikommutantenbildung keine Selbstadjungiertheit herbeiführt.

\subsection{Beispiel: $L^\infty$ ist vNA}

Durch $L^\infty \ni g \mapsto M_g \in \B(L^2)$ wird $L^\infty$ als Operatoralgebra dargestellt, vgl. Borel-FK. Ist $(\Omega,\Sigma,\mu)$ lokalisierbar, so zerlege $L^2(\Omega)$ in direkte Summe $\oplus_i L^2(\Omega_i)$ mit $\mu(\Omega_i) < \infty$. Zeige dann für einen solchen Summanden, dass er seine eigene Kommutante ist, somit vNA.

% TODO Beweis skizzieren. Warum M_f|_{L^\infty} bzgl. L^2-Operatornorm beschränkt?

\section{Tensorprodukte}

\subsection{algebraische Theorie}

Seien $E,F$ Vektorräume, $E^*,F^*$ ihre algebraischen Duale. Bezeichne mit $\Bil(E^*,F^*)$ die bilinearen Funktionale auf $E^*\times F^*$. Für $e\in E, f\in F$ definiere $e\otimes f\in \Bil(E^*, F^*)$ via $e\otimes f(e',f')\coloneqq e'(e)f'(f)$. Es heißt $e\otimes f$ \emph{elementarer Tensor} und $E\otimes F\coloneqq \LH \{e\otimes f: e\in E, f\in F\}$ heißt das \emph{Tensorprodukt} von $E$ mit $F$. Die Zuordnung $i: E\times F \ni (e,f) \mapsto e\otimes f \in E\otimes F$ ist bilinear und aus $0\neq e\in E, 0\neq f\in F$ folgt $e\otimes f \neq 0$, jedoch ist $i$ nicht injektiv. Ist $(e_i)$ Basis von $E$, $(f_j)$ Basis von $F$, dann $\{e_i\otimes f_j: i\in I, j\in J\}$ Basis von $E\otimes F$. Für einen weiteren Vektorraum $W$ gilt die Isomorphie $\Bil(E\times F, W)\cong \Lin(E\otimes F, W)$.

Darstellung eines Tensors ist nicht eindeutig, aber für eine minimale Darstellung (d.h. Anzahl der Summanden ist minimal) $x=\sum_{i=1}^n e_i\otimes f_i$ gilt $\{e_1,\dots,e_n\}$ sowie $\{f_1,\dots,f_n\}$ l.u. (und umgekehrt, vgl. endlich-Rang Operatoren!) und ist $0=\sum_{i=1}^n e_i\otimes f_i$ und $\{e_1,\dots,e_n\}$ l.u., so folgt $f_i=0$ für alle $i$.

\subsubsection{Tensorprodukt von linearen Abbildungen}

Für $A\in \Lin(E,E_1), B\in \Lin(F,F_1)$ gibt es eindeutige (universelle Eigenschaft!) lineare Abbildung $A\boxtimes B: E\otimes F \ni (e,f)\mapsto Ae\otimes Bf \in E_1\otimes F_1$. Wir wollen gerne $A\otimes B$ für $A\boxtimes B$ schreiben. Dazu identifizieren wir $\Lin(E,E_1)\otimes \Lin(F,F_1)$ mit einem Unterraum von $\Lin(E\otimes F, E_1\otimes F_1)$. Wegen $\Lin(E,E_1)\times \Lin(F,F_1) \ni (A,B) \mapsto A\boxtimes B \in \Lin(E\otimes F, E_1\otimes F_1)$ bilinear gibt es $\beta: \Lin(E,E_1)\otimes \Lin(F,F_1) \to \Lin(E\otimes F, E_1\otimes F_1)$ mit $\beta(A\otimes B) = A\boxtimes B$. Man zeigt, dass $\beta$ injektiv ist und kann dann wie gewünscht $A\otimes B$ mit $A\boxtimes B$ identizifieren.

\subsubsection{Tensorprodukt von Algebren}

Sind $\A,\B$ Algebren, so gibt es eindeutige Multiplikation auf $\A\otimes \B$ mit $x\otimes y \cdot x'\otimes y' = xx'\otimes yy'$.

\subsubsection{n-faches Tensorprodukt über Linearformen}

Erhalte Einbettung $E_1\otimes \dots \otimes E_n \hookrightarrow \Mult(E_1^*\times \dots \times E_n^*, \K)$ via $m(e_1\otimes \dots \otimes e_n) \coloneqq ((e_1', \dots, e_n')\mapsto e_1'(e_1)\cdot \dots \cdot e_n'(e_n)$. Dann $E_1\otimes \dots \otimes E_n \cong m(E_1\otimes \dots \otimes E_n)$.

\subsubsection{Tensorprodukt über endlich-Rang Operatoren}

Für $e\in E, f\in F$ definiere $t_{e,f}: E^*\ni e' \mapsto \langle e, e' \rangle f$, also $t_{e,f}\in \Lin(E^*,F)$. Es ist $E\times F \ni (e,f) \mapsto t_{e,f}$ bilinear, also gibt es $\beta: E\otimes F \to \Lin(E^*,F)$ mit $\beta(e\otimes f) = t_{e,f}$, welches injektiv ist, also $E\otimes F \hookrightarrow \Lin(E^*,F)$.

Ist $E\neq E^{**}$, so ist $\beta$ sicher nicht surjektiv!

Der \emph{Rang eines Tensors} $x\in E\otimes F$ ist gegeben durch $\Rang(\beta(x))$ und stimmt mit der Länge einer minimalen Darstellung überein.

Betrachte nun $\beta: E^*\otimes F \to \F(E^{**},F)$. Diese ist nach wie vor injektiv, aber i.A. nicht surjektiv. Wenn wir jedoch $\beta(e'\otimes f)|_{E\subset E^{**}}$ betrachten, so bleibt die Zuordnung injektiv und wird sogar surjektiv, also $E^*\otimes F \cong \F(E,F)$. Wir können also die endlich-Rang Operatoren als Tensorprodukt verstehen! Dies gilt ferner für die topologischen Dualräume.

\subsubsection{Beispiel: Matrizen als TP und ihre Spur}

Betrachte $(\K^n)^* \otimes \K^m$. Es ist $t_{e_j',e_i}=e_{ij}$, also $m\times n$ Matrizen sind Tensorprodukt. Ist $A$ eine $m\times n$-Matrix mit Zeilen $a_1,\dots,a_m \in (\K^n)^*$, so ist $A=\sum_{j=1}^m a_j \otimes e_j$, analog: sind $b_1,\dots,b_n \in \K^m$ die Spalten von $A$, so gilt $A=\sum_{i=1}^n e_i \otimes b_i$. Definiere $(\K^n)^* \times \K^n \ni (x',y)\mapsto \langle y, x' \rangle$ bilinear, diese besitzt Fortsetzung $\tau: (\K^n)^*\otimes \K^n = M_n \to \K$ mit $\tau(A)=\Spur(A)$. Nutze dies später, um den Begriff der Spur zu verallgemeinern!

\end{document}
