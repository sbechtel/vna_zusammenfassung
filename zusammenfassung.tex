\documentclass[11pt,a4paper]{scrartcl}

\usepackage[utf8]{inputenc}
\usepackage[T1]{fontenc}
\usepackage[ngerman]{babel}
\usepackage{amsmath,amsthm,amssymb,dsfont}
\usepackage{mathtools}
\usepackage[paper=a4paper,left=25mm,right=25mm,top=25mm,bottom=25mm]{geometry}
\usepackage{float}
\usepackage{hyperref}
\usepackage{enumerate}
\usepackage{url}
\usepackage{tikz}
\usepackage{tikz-cd}
\usepackage{esint}
\usepackage{csquotes}
\usepackage{textcomp}

\usepackage{setspace}

\parindent 0pt
\linespread{1.5}

% Makros

\newcommand{\N}{\mathbb{N}} % natuerliche Zahlen
\newcommand{\Z}{\mathbb{Z}} % ganze Zahlen
\newcommand{\Q}{\mathbb{Q}} % rationale Zahlen
\newcommand{\R}{\mathbb{R}} % reelle Zahlen
\newcommand{\K}{\mathbb{K}} % Körper
\newcommand{\C}{\mathbb{C}} % komplexe Zahlen
\newcommand{\D}{\mathcal{D}}
\newcommand{\E}{\mathcal{E}}
\newcommand{\Hc}{\mathcal{H}}
\newcommand{\Kc}{\mathcal{K}}
\newcommand{\Sc}{\mathcal{S}}
\newcommand{\A}{\mathcal{A}}
\newcommand{\B}{\mathcal{B}}
\newcommand{\M}{\mathcal{M}}
\newcommand{\Nc}{\mathcal{N}}
\newcommand{\Rc}{\mathcal{R}}
\newcommand{\F}{\mathcal{F}}
\newcommand{\T}{\mathcal{T}}
\newcommand{\Ifin}{\mathrm{I}_\mathrm{endl}}
\newcommand{\Iinf}{\mathrm{I}_\infty}
\newcommand{\IIfin}{\mathrm{II}_1}
\newcommand{\IIinf}{\mathrm{II}_\infty}
\newcommand{\III}{\mathrm{III}}
\newcommand{\circum}{\text{\textasciicircum}}
\renewcommand{\Re}{\mathrm{Re}\,}

% Umgebungen für Definitionen, Sätze, usw.

\theoremstyle{plain}
\newtheorem{thm}{Satz}[section]
\newtheorem*{lem}{Lemma}
\newtheorem{cor}[thm]{Korollar}
\newtheorem{prop}[thm]{Proposition}
\newtheorem*{ex}{Beispiel}
\newtheorem*{ntion}{Notation}

\theoremstyle{definition}
\newtheorem{defn}[thm]{Definition}

\theoremstyle{remark}
\newtheorem*{bem}{Bemerkung}
\newtheorem{bemnumber}[thm]{Bemerkung}

\def\Satzrefname{Satz}

\DeclareMathOperator{\supp}{supp}
\DeclareMathOperator{\esssupp}{ess supp}
\DeclareMathOperator{\id}{id}
\DeclareMathOperator{\loc}{loc}
\DeclareMathOperator{\pv}{pv}
\DeclareMathOperator{\grad}{grad}
\DeclareMathOperator{\Bil}{Bil}
\DeclareMathOperator{\Lin}{Lin}
\DeclareMathOperator{\Mult}{Mult}
\DeclareMathOperator{\LH}{LH}
\DeclareMathOperator{\Rang}{Rang}
\DeclareMathOperator{\Spur}{Spur}
\DeclareMathOperator{\tr}{tr}
\DeclareMathOperator{\vN}{vN}

\begin{document}

\title{Zusammenfassung von Neumann Algebren}
\author{Sebastian Bechtel}
\maketitle

\section{Erster Kontakt}

Eine *-Algebra $\M \subset \B(\Hc)$ heißt \emph{Operatoralgebra}. Ist $\M$ stop-abgeschlossen, so heißt $\M$ \emph{von Neumann Algebra (vNA)}. Durch GNS-Darstellung lässt sich jede C*-Algebra als Operatoralgebra darstellen. Sei $\Sc \subset \B(\Hc)$, da der Schnitt von vNAen wieder vNA ist, existiert kleinste vNA $\vN(\Sc)$, die $\Sc$ enthält. Ist $\Sc$ Operatoralgebra, so gilt $\vN(\Sc)=\overline{\Sc}^{stop}$, dies ist aber nicht offensichtlich, da Involution nicht stop-stetig ist!

Für $\M \subset \B(\Hc)$ ist die \emph{Kommutante} gegeben durch $\M'\coloneqq \{ x\in\B(\Hc): xy=yx \text{ für all } y\in \M \}$. Es gilt immer $\M'$ stop-abgeschlossen und ist $\M$ Operatoralgebra, so auch $\M'$, somit $\M'$ vNA\@. Ferner gilt $1_\Hc\in \M'$ und $\M'=\M'''$ (wegen $\M\subset \M''$), also gilt für $\M$ Operatoralgebra auch nach Bikommutantensatz (vgl.\ später) $\M'$ vNA\@! Aus $\M \subset \M''$ folgt für $\M$ Operatoralgebra, dass $\vN(\M)\subset \M''$. Ist $1_\Hc\in M$, so folgt mit Bikommutantensatz $\M''\subset \vN(\M)''=\vN(\M)$, also $\M''=\vN(\M)$. Ist $\M$ nicht s.a., geht alles schief: Die Matrizen $\left(\begin{smallmatrix} 0 & a \\ 0 & 0 \end{smallmatrix}\right)$ haben als Kommutante $\left(\begin{smallmatrix} b & c \\ 0 & b \end{smallmatrix}\right)$, also ist jene nicht s.a., somit folgt wegen $\M'=\M'''$, dass Bikommutantenbildung keine Selbstadjungiertheit herbeiführt.

\subsection{Beispiel: $L^\infty$ ist vNA}

Durch $L^\infty \ni g \mapsto M_g \in \B(L^2)$ wird $L^\infty$ als Operatoralgebra dargestellt, vgl. Borel-FK\@. Ist $(\Omega,\Sigma,\mu)$ lokalisierbar, so zerlege $L^2(\Omega)$ in direkte Summe $\oplus_i L^2(\Omega_i)$ mit $\mu(\Omega_i) < \infty$. Zeige dann für einen solchen Summanden, dass er seine eigene Kommutante ist, somit vNA: Sei $T\in L^\infty(\Omega)'$ und setze $f\coloneqq T(1)\in L^2(\Omega)$. Es gilt für $g\in L^\infty$ nun $T(g)=TM_g(1)=M_gT(1)=gf=M_f(g)$, also $T|_{L^\infty}=M_f|_{L^\infty}$, somit $M_f: L^\infty \subset L^2 \to L^2$ beschränkt. Wäre $f\not\in L^\infty$, würde es für $n\in \N$ messbare Mengen $\Omega_n \subset \Omega$ geben mit $|f(\omega)| \geq n$ f.ü. auf $\Omega_n$ und $g_n\coloneqq \chi_{\Omega_n}/\mu(\Omega_n)^{1/2}$ würde Beschränktheit auf $L^\infty$ widersprechen, also $f\in L^\infty$ und $M_f=T$ auf dichter Teilmenge, also $T\in L^\infty$.

\section{Tensorprodukte}

\subsection{algebraische Theorie}

Seien $E,F$ Vektorräume, $E^*,F^*$ ihre algebraischen Duale. Bezeichne mit $\Bil(E^*,F^*)$ die bilinearen Funktionale auf $E^*\times F^*$. Für $e\in E, f\in F$ definiere $e\otimes f\in \Bil(E^*, F^*)$ via $e\otimes f(e',f')\coloneqq e'(e)f'(f)$. Es heißt $e\otimes f$ \emph{elementarer Tensor} und $E\otimes F\coloneqq \LH \{e\otimes f: e\in E, f\in F\}$ heißt das \emph{Tensorprodukt} von $E$ mit $F$. Die Zuordnung $i: E\times F \ni (e,f) \mapsto e\otimes f \in E\otimes F$ ist bilinear und aus $0\neq e\in E, 0\neq f\in F$ folgt $e\otimes f \neq 0$, jedoch ist $i$ nicht injektiv. Ist $(e_i)$ Basis von $E$, $(f_j)$ Basis von $F$, dann $\{e_i\otimes f_j: i\in I, j\in J\}$ Basis von $E\otimes F$. Für einen weiteren Vektorraum $W$ gilt die Isomorphie $\Bil(E\times F, W)\cong \Lin(E\otimes F, W)$.

Darstellung eines Tensors ist nicht eindeutig, aber für eine minimale Darstellung (d.h. Anzahl der Summanden ist minimal) $x=\sum_{i=1}^n e_i\otimes f_i$ gilt $\{e_1,\dots,e_n\}$ sowie $\{f_1,\dots,f_n\}$ l.u. (und umgekehrt, vgl.\ endlich-Rang Operatoren!) und ist $0=\sum_{i=1}^n e_i\otimes f_i$ und $\{e_1,\dots,e_n\}$ l.u., so folgt $f_i=0$ für alle $i$.

\subsubsection{Tensorprodukt von linearen Abbildungen}

Für $A\in \Lin(E,E_1), B\in \Lin(F,F_1)$ gibt es eindeutige (universelle Eigenschaft!) lineare Abbildung $A\boxtimes B: E\otimes F \ni (e,f)\mapsto Ae\otimes Bf \in E_1\otimes F_1$. Wir wollen gerne $A\otimes B$ für $A\boxtimes B$ schreiben. Dazu identifizieren wir $\Lin(E,E_1)\otimes \Lin(F,F_1)$ mit einem Unterraum von $\Lin(E\otimes F, E_1\otimes F_1)$. Wegen $\Lin(E,E_1)\times \Lin(F,F_1) \ni (A,B) \mapsto A\boxtimes B \in \Lin(E\otimes F, E_1\otimes F_1)$ bilinear gibt es $\beta: \Lin(E,E_1)\otimes \Lin(F,F_1) \to \Lin(E\otimes F, E_1\otimes F_1)$ mit $\beta(A\otimes B) = A\boxtimes B$. Man zeigt, dass $\beta$ injektiv ist und kann dann wie gewünscht $A\otimes B$ mit $A\boxtimes B$ identizifieren.

\subsubsection{Tensorprodukt von Algebren}

Sind $\A,\B$ Algebren, so gibt es eindeutige Multiplikation auf $\A\otimes \B$ mit $x\otimes y \cdot x'\otimes y' = xx'\otimes yy'$.

\subsubsection{n-faches Tensorprodukt über Linearformen}

Erhalte Einbettung $E_1\otimes \dots \otimes E_n \hookrightarrow \Mult(E_1^*\times \dots \times E_n^*, \K)$ via $m(e_1\otimes \dots \otimes e_n) \coloneqq ((e_1', \dots, e_n')\mapsto e_1'(e_1)\cdot \dots \cdot e_n'(e_n)$. Dann $E_1\otimes \dots \otimes E_n \cong m(E_1\otimes \dots \otimes E_n)$.

\subsubsection{Operatoren endlichen Ranges als Tensorprodukt}

Für $e\in E, f\in F$ definiere $\mathrm{t}_{e,f}: E^*\ni e' \mapsto \langle e, e' \rangle f$, also $\mathrm{t}_{e,f}\in \Lin(E^*,F)$. Es ist $E\times F \ni (e,f) \mapsto \mathrm{t}_{e,f}$ bilinear, also gibt es $\beta: E\otimes F \to \Lin(E^*,F)$ mit $\beta(e\otimes f) = \mathrm{t}_{e,f}$, welches injektiv ist, also $E\otimes F \hookrightarrow \Lin(E^*,F)$.

Ist $E\neq E^{**}$, so ist $\beta$ sicher nicht surjektiv!

Der \emph{Rang eines Tensors} $x\in E\otimes F$ ist gegeben durch $\Rang(\beta(x))$ und stimmt mit der Länge einer minimalen Darstellung überein.

Betrachte nun $\beta: E^*\otimes F \to \F(E^{**},F)$. Diese ist nach wie vor injektiv, aber i.A. nicht surjektiv. Wenn wir jedoch $\beta(e'\otimes f)|_{E\subset E^{**}}$ betrachten, so bleibt die Zuordnung injektiv und wird sogar surjektiv, also $E^*\otimes F \cong \F(E,F)$. Wir können also die endlich-Rang Operatoren als Tensorprodukt verstehen! Dies gilt ferner für die topologischen Dualräume.

\subsubsection{Beispiel: Matrizen als TP und ihre Spur}

Betrachte $(\K^n)^* \otimes \K^m$. Es ist $\mathrm{t}_{e_j',e_i}=e_{ij}$, also $m\times n$ Matrizen sind Tensorprodukt. Ist $A$ eine $m\times n$-Matrix mit Zeilen $a_1,\dots,a_m \in (\K^n)^*$, so ist $A=\sum_{j=1}^m a_j \otimes e_j$, analog: sind $b_1,\dots,b_n \in \K^m$ die Spalten von $A$, so gilt $A=\sum_{i=1}^n e_i \otimes b_i$. Definiere $(\K^n)^* \times \K^n \ni (x',y)\mapsto \langle y, x' \rangle$ bilinear, diese besitzt Fortsetzung $\tau: (\K^n)^*\otimes \K^n = M_n \to \K$ mit $\tau(A)=\Spur(A)$. Nutze dies später, um den Begriff der Spur zu verallgemeinern!

\subsection{topologische Tensorprodukte}

\subsubsection{Tensorprodukte von Hilberträumen}

Bezeichne mit $\odot$ das algebraische Tensorprodukt. Sind $\Hc,\Kc$ Hilberträume, so gibt es auf $\Hc\odot \Kc$ ein eindeutiges Skalarprodukt $\langle \cdot, \cdot \rangle_{HS}$ mit $\langle x\otimes y, x'\otimes y' \rangle_{HS} = \langle x,x' \rangle_\Hc \langle y, y' \rangle_\Kc$. Also ist $(\Hc\otimes \Kc, \langle \cdot, \cdot \rangle_{HS})$ Prähilbertraum. Dessen Vervollständigung heißt \emph{Hilbertraumtensorprodukt} von $\Hc$ und $\Kc$, schreibe $\Hc \otimes_{HS} \Kc$ bzw. $\Hc \otimes \Kc$. Die induzierte Norm $\|\cdot\|_{HS}$ ist eine \emph{Kreuznorn}, d.h. $\|x\otimes y\|_{HS}=\|x\|_\Hc \|y\|_\Kc$ (somit ist $B: \Hc \times \Kc \to W$ bilinear genau dann beschränkt, wenn es $T_B: \Hc \odot \Kc \to W$ ist, also erhalten wir eine topologische Version der universellen Eigenschaft für $\Hc \otimes \Kc$). Außerdem folgt aus $x_i \to x, y_j\to y$, dass $x_i \otimes y_j \to x\otimes y$ und sind $(e_i), (f_j)$ ONBs von $\Hc$ bzw. $\Kc$, so ist $\{ e_i \otimes f_j: i\in I, j\in J \}$ ONB von $\Hc\otimes \Kc$.

Beispiele: 1) Es gilt $\ell^2(I)\otimes \ell^2(J) = \ell^2(I\times J)$ (Dimensionsvergleich!) 2) Sind $(\Omega_1, \Sigma_1,\mu_1), (\Omega_2,\Sigma_2,\mu_2)$ $\sigma$-endliche Maßräume, dann gilt $L^2(\Omega_1\times \Omega_2) \cong L^2(\Omega_1)\otimes L^2(\Omega_2)$ und der unitäre Operator ist eindeutig mit $U(f_1\otimes f_2) = f_1(\omega_1)f_2(\omega_2)$ (nutze topologische universelle Eigenschaft!)

Sind $A\in \B(\Hc), B\in \B(\Kc)$, denn gibt es eindeutigen Operator $A\otimes B\in \B(\Hc\otimes \Kc)$ mit $A\otimes B(x\otimes y)=Ax\otimes By$ und es gilt $\|A\otimes B\|=\|A\|\|B\|$. Eindeutigkeit folgt aus der Eindeutigkeit auf dem dichten algebraischen Tensorprodukt. Zeige $\|A\otimes B\|=\|A\|\|B\|$ auf $\Hc \odot \Kc$, dann stetige Fortsetzung. Nutze $A\otimes B=(A\otimes 1_\Kc)(1_\Hc\otimes B)$ (denn $((A\otimes 1_\Kc)(1_\Hc\otimes B))(x\otimes y)=A\otimes 1_\Kc(x\otimes By)=Ax\otimes By$) und $\|A\otimes 1_\Kc\| \leq \|A\|$. Außerdem gilt $(A\otimes B)^*=A^*\otimes B^*$.

\subsection{Tensorprodukte auf Banachräumen}

Seien $E,F$ Banachräume. Auf $E\odot F$ wird durch $\|z\|_\pi \coloneqq \inf \{ \sum_i \|x_i\|_E \|y_i\|_F: z = \sum_i x_i \otimes y_i \}$ eine Kreuznorm definiert, welche \emph{projektive Norm} oder \emph{maximale Norm} heißt. Motivation: Für eine Kreuznorm $\|\cdot\|$ gilt $\|z\|\leq \sum_i \|x_i\|\|y_i\|$ wegen Dreiecksungleichung und Kreuznormeigenschaft. Beachte, dass Supremum keinen Sinn machen würde, da wegen $0=x\otimes y + (-x)\otimes y$ für $x,y\neq 0$ bereits die Definitheit verletzt wäre.

Wiederum auf $E\odot F$ definieren wir eine weiter Norm, genannt \emph{injektive Norm} oder \emph{$\varepsilon$-Tensornorm}, durch $\|z\|_\varepsilon \coloneqq \sup \{ |e'\otimes f'(z)|: e'\in E^*_1, f'\in F^*_1 \}$. Die Kreuznormeigenschaft folgt aus Hahn-Banach: zu $x\in E, y\in F$ gibt es $e'\in E^*_1, f'\in F^*_1$ mit $e'(x)=\|x\|_E, f'(y)=\|y\|_F$, also $\|x\otimes y\|_\varepsilon \geq |e'\otimes f'(x\otimes y)|=\|x\|_E \|y\|_F$.

\section{Die vNA $\B(\Hc)$}

\subsection{Projektionen}

Ein $p\in \B(\Hc)$ heißt \emph{Projektion}, falls $p^2=p$ gilt und \emph{orthogonale Projektion}, falls $p$ Projektion und $p^*=p$. Man nennt $v\in \B(\Hc)$ eine \emph{partielle Isometrie}, falls $v^*v$ eine orthogonale Projektion ist (automatisch s.a., Projektion ist zu prüfen). Ist $v$ partielle Isometrie so ist $\Nc(v)^\perp$ der Anfangsraum von $v$ und $\overline{\Rc(v)}$ der Zielraum von $v$. Es ist $v^*v$ orthogonale Projektion auf den Anfangsraum, genannt \emph{initiale Projektion}, und $vv^*$ ist orthogonale Projektion auf den Zielraum, genannt \emph{finale Projektion}. Zwischen Anfangsraum und Zielraum ist eine partielle Isometrie eine Isometrie.

Für $x\in \B(\Hc)$ definieren wir Anfangsraum und Zielraum wie oben und bezeichnen die initiale Projektion als \emph{Rechtsträger} $\mathrm{s}_r(x)$ und die finale Projektion als \emph{Linksträger} $\mathrm{s}_l(x)$. Für $x$ s.a.\ definiere den Träger $s(x)\coloneqq \mathrm{s}_l(x)=\mathrm{s}_r(x)$ (wegen $\Nc(x)^\perp=\overline{\Rc(x^*)}=\overline{\Rc(x)}$). Es gilt (analog zu den partiellen Isometrien!) $\mathrm{s}_l(x)=\mathrm{s}(xx^*)$ und $\mathrm{s}_r(x)=s(x^*x)$. Außerdem gilt $\mathrm{s}_r(x)=\mathrm{s}_r(|x|)$ (nutze $\ker(x)=\ker(x^*x)$).

\subsubsection{polare Zerlegung}

Eine komplexe Zahl $z$ lässt sich schreiben als $z=e^{i\varphi}|z|$, also $|z| \geq 0$ und

% TODO polare Zerlegung fertig schreiben. Motivation/Analogie zu C: MSE-Thread

\subsection{Die Spur auf $\B(\Hc)$ (vgl. Ü.A. 34, 37)}

Sei $(e_i)$ ONB von $\C^n$ und $T\in \B(\C^n)$ mit Darstellungsmatrix $A$. Dann definiert man $\Spur(A)=\sum_i a_{ii}$. Nun gilt aber $\sum_i a_{ii} = \sum_i \langle Te_i, e_i \rangle$. Diese Formel wollen wir auf beliebige Hilberträume verallgemeinern!

Nun sei also $\Hc$ wieder beliebiger Hilbertraum und $(e_i)$ ONB von $\Hc$. Ist $0\leq x\in \B(\Hc)$, so ist $\tr(x)\coloneqq \sum_i \langle x e_i, e_i \rangle \in [0,\infty]$ definiert. Ist $x\in \B(\Hc)$, so gilt $\tr(x^*x)=\tr(xx^*)$ wegen $\langle x^*x e_i, e_i \rangle = \langle e_i, xx^* e_i \rangle = \overline{\langle xx^* e_i, e_i \rangle} = \langle xx^* e_i, e_i \rangle$, also für $0 \leq x$ und $u$ unitär: $\tr(u^*xu)=\tr(u^*x^{1/2}x^{1/2}u)=\tr((x^{1/2}u)^*x^{1/2}u)=\tr(x^{1/2}uu^*x^{1/2})=\tr(x)$, somit ist $\tr$ für positive Operatoren basisunabhängig!

Ist nun $x\in \F(\Hc)$, so können wir $x$ als Linearkombination positiver Endlichrangoperatoren schreiben ($\Re x = 1/2(x+x^*) \in \F(\Hc)$ und dann durch Einteilung der Eigenwerte in positive und negative). Existenz der Reihe ist dann klar und wegen Basisunabhängigkeit entspricht $\tr$ der Summe der endlich vielen Eigenwerte, also $\tr(x) < \infty$.

Die Basisunabhängigkeit überträgt sich ebenso auf den Fall von Endlichrangoperatoren. Ist $x\in \F(\Hc)$ und $y\in \B(\Hc)$ (oBdA unitär), so gilt $\tr(xy)=\tr(yx)$, denn $yx\in \F(\Hc)$ und $\tr(xy)=\tr(y^*yxy)=\tr(yx)$.

\subsubsection{Bezug zu reinen Zuständen auf $M_n$ (vgl. Ü.A. 58 Spektraltheorie)}

Es ist $M_n\times M_n \ni (x,y) \mapsto \tr(y^*x)$ ein Skalarprodukt auf $M_n$, aus Riesz-Frechet folgt, dass ein Funktional $\varphi$ auf $M_n$ durch eine Dichtematrix $\Phi$ via $\varphi(x)=\tr(\Phi x)$ dargestellt werden kann. Ist $\Phi \geq 0$ und $\tr(\Phi)=1$, so ist $\varphi$ ein Zustand. Dieser ist genau dann rein, wenn $\Phi$ eine eindimensionale orthogonale Projektion ist.

Man rechnet leicht nach, dass $\tr(x \mathrm{t}_{\xi,\eta})=\tr(\mathrm{t}_{\xi,x\eta})=\langle x\eta, \xi \rangle$ gilt. Ist also $\Phi=\mathrm{t}_{\xi,\xi}$ eindimensionale orthogonale Projektion und $x\in M_n$, so gilt $\varphi(x)=\tr(\Phi x)=\tr(x \Phi)=\langle x\xi, \xi \rangle$, also ist $\varphi$ Vektorzustand!

\section{Wichtige Operatorklassen und deren Konstruktion über Tensorprodukte}

Wir hatten gesehen, dass $E^*\odot E \cong \F(E)$ gilt. Ist $E=\Hc$ ein Hilbertraum, so würden wir gerne statt $\odot$ auch $\otimes_{HS}$ bilden können (dies wird auf die Klasse der Hilbert-Schmidt-Operatoren führen). Allerdings ist $\Hc^*$ kein Hilbertraum, wenn man durch die Riesz-Frechet-Identifikation das Skalarprodukt von $\Hc$ zurückziehen will. Wir statten deshalb $\Hc^*$ mit einer neuen Skalarmultiplikation und einem geeigneten Skalarprodukt aus, um so $\Hc^*$ zu einem Hilbertraum zu machen, mit dem $\Hc^*\times \Hc \ni (\xi,\eta) \mapsto \mathrm{t}_{\xi,\eta}$ bilinear ist.

\subsection{Operatoren endlichen Ranges}

Ein Operator $x\in \B(\Hc,\Kc)$ ist genau dann ein Operator endlichen Ranges, $x\in \F(\Hc,\Kc)$, wenn es ONSe $\{e_1,\dots,e_n\}$ in $\Hc$, $\{f_1,\dots,f_n\}$ von $\Kc$ gibt mit $x=\sum_i \lambda_i \mathrm{t}_{e_i,f_i}$. Nutze dazu die polare Zerlegung $x=v|x|$, dann ist $|x|$ ebenso Endlichrangoperator und kann nach dem Spektralsatz der linearen Algebra als $|x|=\sum_i \lambda_i \mathrm{t}_{e_i,e_i}$ geschrieben werden. Dann erhält man die gewünschte Darstellung, wenn man das zweite ONS via $f_i\coloneqq ve_i$ definiert. Aus dieser Äquivalenz ist wegen $\mathrm{t}_{\xi,\eta}^*=\mathrm{t}_{\eta,\xi}$ auch klar, dass $x\in\F(\Hc,\Kc)$ genau dann, wenn $x^*\in\F(\Kc,\Hc)$.

Daraus folgt auch die sogenannte \emph{Hilbert-Schmidt-Darstellung} eines Tensors: Ist $x=\sum_i \xi_i\otimes \eta_i \in \Hc\odot \Kc$, so gibt es ONSe $\{e_1,\dots,e_n\}$ von $\Hc$, $\{f_1,\dots,f_n\}$ von $\Kc$ sowie $\lambda_1,\dots,\lambda_n > 0$ mit $x=\sum_i \lambda_i e_i\otimes f_i$.

\subsection{kompakte Operatoren}

Für $\xi\in \Hc\odot \Kc$ gilt $\|\xi\|_\varepsilon = \|\beta(\xi)\|_\mathrm{op}$. Somit folgt $\Hc \otimes_\varepsilon \Hc \cong K(\Hc)$.

\subsection{Hilbert-Schmidt-Operatoren}

Sind $\xi,\eta\in \Hc^*\odot \Hc$ und seien $x\coloneqq \beta(\xi), y\coloneqq \beta(\eta)$, so gilt $\langle \xi, \eta \rangle_\mathrm{HS} = \tr(y^*x)$. Definiere daher via $\|x\|_2 \coloneqq \tr(x^*x)^{1/2}$ eine Norm auf $\F(\Hc)$, genannt \emph{Hilbert-Schmidt-Norm}. 

Für eine Darstellung $x=\sum_i \lambda_i \mathrm{t}_{e_i,f_i}$, wobei $(e_i),(f_i)$ ONSe, ist die Hilbert-Schmidt-Norm gegeben durch $\|x\|_2 = \left(\sum_i |\lambda_i|^2\right)^{1/2}$. Daraus folgt auch, dass $\|\beta(\xi)\|_\mathrm{op}\leq \|\beta(\xi)\|_2=\|\xi\|_\mathrm{HS}$: Es gilt $x^*x=\sum_i |\lambda_i|^2 \mathrm{t}_{e_i,e_i}$. Sei $k$ so gewählt, dass $\lambda_k$ maximaler Eigenwert von $x^*x$ ist. Dann folgt mit C*-Eigenschaft: $\|x\|_\mathrm{op}^2=\|x^*x\|_\mathrm{op} = |\lambda_k|^2 \leq \sum_i |\lambda_i|^2 = \|x\|_2^2$.

Nun können wir die \emph{Hilbert-Schmidt-Operatoren} definieren. Sei $\beta: \Hc^*\otimes_\mathrm{HS} \Hc \to (K(H), \|\cdot\|_\mathrm{op})$ die stetige Fortsetzung von $\beta$. Dann sind die Hilbert-Schmidt-Operatoren gegeben durch $\mathrm{HS}(\Hc) \coloneqq \beta(\Hc^*\otimes_\mathrm{HS}\Hc) \subset K(\Hc)$. 

Es ist $x\in \mathrm{HS}(\Hc)$ genau dann, wenn $\tr(x^*x)<\infty$ (vgl. $\ell^2$!) und dann gibt es eine Darstellung $x=\sum_i \lambda_i \mathrm{t}_{e_i,f_i}$ mit $\lambda_i > 0$, $\sum_i |\lambda_i|^2 < \infty$ und $(e_i)_i, (f_i)_i$ ONSe.

Beispiel: Sei $K\in L^2(\Omega\times \Omega)$ Integralkern, dann ist $T_K: L^2(\Omega) \ni f \mapsto \int_\Omega K(\omega,\omega')f(\omega')\mathrm{d}\omega' \in L^2(\Omega)$ in $\mathrm{HS}(\Hc)$.

\subsection{Spurklasseoperatoren}

Für $\xi\in \Hc^*\odot \Hc$ in Hilbert-Schmidt-Darstellung $\xi=\sum_{i=i}^n \lambda_i e_i\otimes f_i$ gilt $\|\xi\|_\pi = \sum_{i=1}^n \lambda_i$ (also entspricht  die $\pi$-Norm der $l^1$-Norm!) und es lässt sich damit $\|\xi\|_\pi = \tr(|x|)$ zeigen (sieht ebenfalls aus wie $l^1$-Norm). Also ist $x\mapsto \tr(|x|)$ Norm auf $\F(\Hc)$.

Da $\|\beta(\xi)\|_\mathrm{op} = \|\xi\|_\varepsilon \leq \|\xi\|_\pi$, gibt es stetige Fortsetzung $\beta: \Hc^*\otimes_\pi \Hc \to (K(\Hc),\|\cdot\|_\mathrm{op})$ und wir definieren die \emph{Spurklasseoperatoren} via $\T(\Hc)\coloneqq \beta(\Hc^*\otimes_\pi \Hc)$.

Es ist $x\in \T(\Hc)$ genau dann, wenn $\tr(|x|)<\infty$ und in diesem Fall gibt es eine Darstellung $x=\sum_i \lambda_i \mathrm{t}_{e_i,f_i}$ wobei $\lambda_i > 0$, $(\lambda_i)_{i\in I}\in l^1(I)$ und $(e_i)_i, (f_i)_i$ ONSe.

Es ist $\T(\Hc)\subset \B(\Hc)$ ein zweiseitiges *-Ideal (gleich wichtig für die Dualitätstheorie!).  

\subsection{Dualitätstheorie}

Wir haben gesehen, dass es eine Analogie zwischen Folgenräumen und Operatorenräumen gibt, und zwar entsprechen die endlichen Folgen $c_{00}$ den Operatoren endlichen Ranges $\F(\Hc)$, die Nullfolgen $c_0$ entsprechen den kompakten Operatoren $K(\Hc)$, die summierbaren Folgen $l^1$ den Spurklasseoperatoren $\T(\Hc)$, die quadratintegrierbaren Folgen $\ell^2$ den Hilbert-Schmidt-Operatoren $\mathrm{HS}(\Hc)$ und die beschränkten Folgen $l^\infty$ den beschränkten Operatoren $\B(\Hc)$.

Die Folgenräume $c_0$, $l^1$ und $l^\infty$ bilden eine Dualitätskette. Diese überträgt sich auf die Operatorklassen, es sind also die Spurklasseoperatoren der Dual von den kompakten Operatoren und die beschränkten Operatoren der Dual der Spurklasseoperatoren.

\subsubsection{Der Dual von $K(\Hc)$}

Für $\Phi\in \T(\Hc)$ und $x\in K(\Hc)$ gilt $\Phi x\in \T(\Hc)$ (Ideal), also ist $K(\Hc) \ni x\mapsto \tr(\Phi x)$ wohldefiniert und stetig. Ist umgekehrt $\varphi\in K(\Hc)^*$, so gibt es $\Phi\in \T(\Hc)$ mit $\varphi(x)=\tr(\Phi x)$. Idee: Betrachte $\varphi|_\mathrm{HS}$. Nach Riesz-Frechet gibt es $\Phi\in \mathrm{HS}(\Hc)$ mit $\varphi(x)=\langle x, \Phi \rangle_\mathrm{HS} = \tr(\Phi^* x)$. Zeige dann: $\Phi^* \in \T(\Hc)$.

Die Zuordnung ist isometrisch, also $K(\Hc)^*\cong \T(\Hc)$.

Außerdem lässt sich $\varphi$ schreiben als $\varphi(x)=\sum_i \lambda_i \langle x f_i, e_i \rangle$ mit $\lambda_i > 0$, $\sum_i \lambda_i < \infty$ und $(e_i),(f_i)$ ONSe.

\subsubsection{Der Dual von $\T(\Hc)$}

Wie oben ist für $x\in \B(\Hc)$ ein stetiges Funktional gegeben durch $\T(\Hc) \ni \Phi \mapsto \tr(\Phi x)$ (wohldefiniert wie oben und Konvergenz in $\|\cdot\|_1$ impliziert insbesondere Konvergenz in $\|\cdot\|_\mathrm{op}$). Umgekehrt gibt es zu $\varphi\in \T(\Hc)^*$ ein $x\in \B(\Hc)$ mit $\varphi(\Phi)=\tr(\Phi x)$. Beweisidee: Ziehe Funktional auf $\Hc^*\otimes_\pi \Hc$ zurück, diese liefert eine beschränkte Bilinearform auf $\Hc^* \times \Hc$, welche wiederum einer Sesquilinearform auf $\Hc\times \Hc$ entspricht. Diese wird durch einen beschränkten Operator eindeutig dargestellt.

Die Zuordnung ist wieder eindeutig und isometrisch, also $\T(\Hc)^*\cong \B(\Hc)$.

\subsection{normale Linearformen}

Ein Funktional $\varphi: \B(\Hc)\to \C$ heißt \emph{normal}, falls es schwach-*-stetig ist, also: Ist $x_i$ Netz in $\B(\Hc)$ mit $\tr((x_i-x)\Phi)\to 0$ für ein $x\in \B(\Hc)$ und alle $\Phi\in \T(\Hc)$, so folgt $\varphi(x_i)\to\varphi(x)$.

Dies ist genau dann der Fall, wenn es $\Phi\in \T(\Hc)$ gibt mit $\varphi(x)=\tr(\Phi x)$.

\section{Topologien auf $\B(\Hc)$}

Bisher kennen wir die Normtopologie, starke Operatortopologie ($\mathrm{stop}$) und schwache Operatortopologie ($\mathrm{swop}$) auf $\B(\Hc)$. Im Folgenden werden wir weitere Topologien kennenlernen und zeigen, dass diese in gewissen Fällen übereinstimmen bzw.\ deren Dualräume übereinstimmen.

\subsection{$\mathrm{swop}=\sigma(\B(\Hc),\F(\Hc))$}

Sei $x_i$ in $\B(\Hc)$. Gilt $\mathrm{swop}-\lim_i x_i = 0$, so auch $|\tr(\sum_{j=1}^n \mathrm{t}_{\xi_j, \eta_j} x_i)|\leq \sum_{j=1}^n |\langle x_i \eta_j, \xi_j \rangle| \to 0$. Umgekehrt zeigt $\tr(\mathrm{t}_{\eta,\xi}x)=\langle x\xi, \eta \rangle$ die $\mathrm{swop}$-Konvergenz.

\subsection{$\mathrm{stop}$-stetige Funktionale sind $\mathrm{swop}$-stetig}

Für $\varphi\in \B(\Hc)^*$ sind $\mathrm{stop}$- und $\mathrm{swop}$-Stetigkeit äquivalent und es gibt eine Darstellung $\varphi(x)=\tr(\Phi x)$ für $\Phi\in \F(\Hc)$, also $(\B(\Hc),\mathrm{stop})^*=\F(\Hc)=(\B(\Hc),\mathrm{swop})^*$.

Das aus $\mathrm{swop}$-Stetigkeit auch $\mathrm{stop}$-Stetigkeit folgt, sowie Funktionale dieser Form $\mathrm{swop}$-stetig sind, ist klar. Wir zeigen, dass ein $\mathrm{stop}$-stetiges Funktional eine solche Darstellung besitzt. Dazu nutzen wir einen Amplifikationstrick!

Aus der $\mathrm{stop}$-Stetigkeit von $\varphi$ folgt, dass $\varphi$ durch endlich viele $\mathrm{stop}$-Halbnormen dominiert wird, also $|\varphi(x)|\leq \left(\sum_{i=1}^n \|x\xi_i\|\right)^{1/2}$. Zu $\hat\Hc\coloneqq \oplus_{i=1}^n \Hc$ betrachten wir den Teilraum $\tilde\Hc \coloneqq \{ \tilde x\tilde \xi = x\xi_1\oplus \dots \oplus x\xi_n: x\in \B(\Hc) \}$. Es gilt wegen oben $|\varphi(x)|\leq \|\tilde x\tilde \xi\|$ und daraus folgt außerdem die Wohldefiniertheit von $\tilde x\tilde \xi \mapsto \varphi(x)$. Also ist $\varphi: \tilde \Hc \to \C$ ein stetiges Funktional, nach Riesz-Frechet gibt es $\tilde \eta$ mit $\varphi(x)=\langle \tilde x\tilde \xi, \tilde \eta \rangle = \sum_{i=1}^n \langle x \xi_i, \eta_i \rangle$.

\subsection{$\sigma-\mathrm{swop}$-Topolgie und $\sigma-\mathrm{stop}$-Topologie}

Sind $(\xi_i)_{i=1,\dots,n},(\eta_i)_{i=1,\dots,n}$ endliche Folgen in $\Hc$, so definiert die Familie der Halbnormen $x\mapsto |\sum_{i=1}^n \langle x\xi_i, \eta_i \rangle|$ die $\mathrm{stop}$-Topologie auf $\B(\Hc)$. Ersetzen wir die endlichen Folgen durch Folgen $(\xi_n),(\eta_n)$ mit $\sum_n \|\xi_n\|^2 < \infty, \sum_n \|\eta_n\|^2 < \infty$, erhalten wir die $\sigma-\mathrm{swop}$-Topologie, welche offensichtlich feiner als die $\mathrm{stop}$-Topologie ist.

Die $\sigma-\mathrm{swop}$-Topologie stimmt mit $\sigma(\B(\Hc),\T(\Hc)$ überein, wird also von den normalen Linearformen induziert (es reichen sogar die normalen Zustände, da normale Linearformen positive Linearkombination von normalen Zuständen sind).

Ebenso definiert man die $\sigma-\mathrm{stop}$ Halbnormen über eine Folge $(\xi_n)$ mit $\sum_n \|\xi\|^2 < \infty$ und diese Topologie wir auch durch die Halbnormen $x\mapsto \varphi(x^*x)^{1/2}$ erzeugt, wobei $\varphi$ normaler Zustand.

Für die Dualräume gilt $(\B(\Hc),\sigma-\mathrm{stop})^*=\T(\Hc)=(\B(\Hc),\sigma-\mathrm{swop})^*$, der Beweis geht wie im Fall von $\mathrm{stop}$- und $\mathrm{swop}$-Topologie mit Amplifikationstrick.

\subsection{Vergleich der Topologien}

Auf $\B(\Hc)_1$ stimmen $\mathrm{swop}$ und $\sigma-\mathrm{swop}$ überein: Die Einheitskugel ist $\sigma-\mathrm{swop}=\sigma^*$ kompakt (Banach-Alaoglu), $\mathrm{swop}$ ist Hausdorff und $\id: (\B(\Hc)_1, \sigma-\mathrm{swop})\to (\B(\Hc)_1,\mathrm{swop})$ ist stetig, also Homöomorphismus.

Außerdem stimmen $\mathrm{stop}$ und $\sigma-\mathrm{stop}$ auf $\B(\Hc)_1$ überein, nutze dazu $\mathrm{stop}-\lim_i x_i = 0$ gdw. $\mathrm{swap}-\lim_i x^*x = 0$ (ebenso mit $\sigma$-Versionen).

\section{Eigenschaften von vNAen}

\subsection{Bikommutantensatz}

Sei $\M \subseteq \B(\Hc)$ Operatoralgebra mit $1_\Hc \in \M$. Dann sind für $x\in \B(\Hc)$ äquivalent: 1) $x\in \overline{\M}^\mathrm{swop}$ 2) $x\in \overline{\M}^\mathrm{stop}$ 3) $x\in \M''$. Beweis: Die Äquivalenz von 1) und 2) ist klar, da $\M$ konvex ist und beide Topologien die selben Duale erzeugen. Da $\M \subseteq \M''$ und $\M''$ automatisch $\mathrm{stop}$-abg.\ ist, folgt aus 2) auch 3). Für 3) nach 2) zeigt man zuerst für $x''\in \M''$ und $\xi\in \Hc$, dass man $x\in \M$ mit $\|(x-x'')\xi\| < \varepsilon$ findet. Durch Amplifikation zeigt man dann, dass auch der Schnitt mit einer Basis der lokalkonvexen Topologie nicht leer ist.

Insbesondere folgt daraus: Eine Operatoralgebra $\M\subseteq \B(\Hc)$ mit $1_\Hc\in \M$ sind äquivalent: 1) $\M$ vNA 2) $\M$ ist $\mathrm{swop}$-abgeschlossen 3) $\M=\M''$.

Außerdem folgt $\overline{\A}^\mathrm{stop}=\overline{\A}^\mathrm{swop}=\A''$, also ist der $\mathrm{stop}$-Abschluss einer C*-Algebra eine vNA (obwohl apriori nicht mal klar ist, dass jener eine Algebra ist, da die Multiplikation auf $\B(\Hc)\times \B(\Hc)$ weder $\mathrm{stop}$-, noch $\mathrm{swop}$-stetig ist).

Gegenbeispiele: 1) Ist $\M$ vNA, so auch auf einem größeren Hilbertraum, aber dann ist $1_\Hc$ nicht in $\M$, jedoch ist die Eins immer in der Kommutante, also $\M\neq\M''$ 2) Die Kommutante von $\{\left(\begin{smallmatrix} a & b \\ 0 & c \end{smallmatrix}\right)\}$ ist $\C$, also ist die Bikommutante $\mathrm{M}_2$.

\subsection{Operatormatrizen und Amplifikation}

Ist $\M$ Operatoralgebra mit $1_\Hc\in \M$, so ist $\vN(\M)\coloneqq \M''$ \emph{erzeugte vNA}. Ist $\Nc$ weitere Operatoralgebra, so ist $\vN(\M,\Nc)\coloneqq \vN(\M\cup\Nc)$.

Sind $\M,\Nc$ vNAen, dann definiere $\M\odot\Nc\coloneqq \LH \{x\otimes y \in \B(\Hc\otimes \Kc): x\in \M, y\in \Nc \}$. Das \emph{von Neumann Tensorprodukt} ist dann $\M\otimes\Nc \coloneqq \vN(\M\odot \Nc)$.

Für $\Kc = \ell^2(I)$ und $\Nc=1_\Kc\coloneqq \C 1_\Kc$ ist $\M\odot 1_\Kc = \{ x\otimes 1_\Kc \in \B(\oplus_i \Hc): x\in \M \}=\{ \left( \begin{smallmatrix} \ddots & & \\ & x & \\ & & \ddots \end{smallmatrix} \right) \}$. Diese Algebra entspricht seiner Bikommutante, also $\M\otimes 1_\Kc = \M \odot 1_\Kc$.

Es gilt $(\M\otimes 1_\Kc)'=\mathrm{M}_I(\M')$. Außerdem gilt $\M\otimes \B(\Kc)=\mathrm{M}_I(\M)$, somit auch $(M\otimes 1_\Kc)=\M'\otimes \B(\Kc)$.

\subsection{Spektraltheorie in vNAen}

Ist $\M$ vNA mit $1_\Hc \in \M$ und $x\in \M$, so besitzt $x$ polare Zerlegung $x=v|x|$ mit $v,x,\mathrm{s}_l(x),\mathrm{s}_r(x)\in \M$ (nutze Bikommutantensatz). Aus $|x|=v^*x$ folgt dann, dass zweiseitige Ideale bereits *-Ideale sind ($x^*=|x|v^*$ und $|x|=v^*x$ im Ideal, da $x$ im Ideal).

Außerdem liegen alle Spektralprojektionen eines $x=x^*\in \M$ wieder in $\M$ (denn $\chi_{(-\infty, \lambda]}(x)=\mathrm{s}((x-\lambda)_{-})$ liegt in $\M$, da $\mathrm{C}^*(1_\Hc, x) \subset \M$ und Träger enthalten sind), ebenso monotone Limiten (da diese $\mathrm{stop}$-konvergieren und $\M$ diesbezüglich abgeschlossen ist).

\subsection{Einselement}

Ist $\M$ vNA auf $\Hc$, so besitzt $\M$ ein Einselement $e$ ($\mathrm{stop}$-Limes einer approximativen Eins, die per Konstruktion monoton und beschränkt ist!). Ist $\M$ nicht degeneriert, also $[\M\Hc]=\Hc$, dann $e=1_\Hc$, sonst $\M=\M|_{[\M\Hc]}\oplus \C (1_\Hc-e)$.

\subsection{Abgeschlossenheit einer vNA}

Eine vNA $\M$ ist $\mathrm{stop}-,\mathrm{swop}-,\sigma-\mathrm{swop}-$ und $\sigma-\mathrm{stop}$-abgeschlossen: Die erste und letzte Implikation gilt, weil die Topologien die gleichen Duale erzeugen und $\M$ konvex ist, die mittlere, weil $\mathrm{swop}$ gröber als $\sigma-\mathrm{swop}$ ist.

\subsection{Prädual einer vNA}

Jede vNA $\M$ hat einen Prädual und dieser ist gegen durch $\T(\Hc)/\{\Phi\in \T(\Hc): \tr(\Phi x)|_\M = 0 \}$.

\subsection{Stetigkeit von Funktionalen}

Mit einem Resultat aus der FA kann man zeigen, dass $\varphi \in \M^*$ genau dann $\sigma-\mathrm{swop}$-stetig (also $\varphi\in \M_*$) ist, wenn $\varphi|_{\M_1}$ bzgl. $\mathrm{swop}$ stetig ist (es gilt die analoge Aussage mit $\mathrm{stop}$).

\subsection{Dichtesatz von Kaplansky}

Ist $\M$ vNA und $\A$ eine $\mathrm{stop}$- bzw. $\mathrm{swop}$-dichte Operatoralgebra, so gilt 1) $\A_\mathrm{h}$ ist $\mathrm{stop}$-dicht in $\M_\mathrm{h}$ 2) $\A_1^\mathrm{h}$ ist $\mathrm{stop}$-dicht in $\M_1^\mathrm{h}$ 3) $\A_1^+$ ist $\mathrm{stop}$-dicht in $\M_1^+$ 4) $\A_1$ ist $\mathrm{stop}$-dicht in $\M_1$.

Beweis: TODO!!!

Dieser Satz ist nützlich, da auf Schnitten mit der Einheitskugel z.B. $\mathrm{stop}$ und $\sigma-\mathrm{stop}$ übereinstimmen, es ist also Spiel mit den Topologien möglich!

Anwendung: Operatoralgebra $\M$ ist vNA gdw. $\M$ bzgl. $\sigma-\mathrm{swop}$ abgeschlossen ist gdw. $\M_1$ bzgl. $\sigma-\mathrm{swop}$ kompakt ist! Die erste Implikation ist bekannt, die Zweite ist Banach-Alaoglu, die Dritte nutzt Kaplansky und das übereinstimmen von $\mathrm{stop}$ und $\sigma-\mathrm{swop}$ auf $\M_1$!

\subsection{topologische Charakterisierung von vNAen}

Eine Operatoralgebra $\M$ ist vNA gdw. $\M$ bzgl. $\sigma-\mathrm{swop}$ abgeschlossen ist gdw. $\M_1$ schwach-*-kompakt ist.

Erste Implikation ist bekannt, $\M_1=\M\cap \B(\Hc)_1$ ist $\sigma-\mathrm{swop}$ abgeschlossen und damit kompakt (Banach-Alaoglu) und aus der Abgeschlossenheit folgt mit Kaplansky insbesondere, dass $\sigma-\mathrm{swop}$ Grenzwerte aus der Einheitskugel im $\mathrm{stop}$-Abschluss liegen, da auf der Einheitskugel die Topologien übereinstimmen.

\subsection{Morphismen}

Die "richtigen" Morphismen sind schwach-*-stetige *-Homomorphismen. Es gilt folgendes Stetigkeitsdiagramm:

\begin{center}\begin{tikzcd}
    \mathrm{stop} \arrow[r, Leftrightarrow] & \mathrm{swop} \arrow[d, Rightarrow]  \\
    \sigma-\mathrm{stop} \arrow[r, Leftrightarrow]  & \sigma-\mathrm{swop},
\end{tikzcd}\end{center}

d.h. $\sigma-\mathrm{swop}$ ist der allgemeinste Stetigkeitsbegriff. Das Bild einer vNA unter einem stetigen *-Homomorphismus ist wieder eine vNA (nutze $\pi(\M_1)=\pi(\M)_1$ ist schwach-*-kompakt)!

\section{Klassifikation von vNAen}

Zur Klassifikation von vNAen werden wir versuchen die Faktoren (einfache Elemente in der Kategorie der vNAen!) zu verstehen und diese zu klassifizieren. Dabei wird die Untersuchung und der Vergleich von Projektionen eine zentrale Rolle einnehmen.

\subsection{Faktoren}

Ist $\M$ vNA, so heißt $\mathrm{Z}(\M)=\{x\in \M: xy=yx \text{ für all } y\in \M\}=\M\cap \M'$ das \emph{Zentrum} von $\M$. Man nennt $\M$ \emph{Faktor}, falls $\mathrm{Z}(\M)=\C$.

Für $\mathfrak{m}\subset\M$ ist genau dann $\mathrm{swop}$-abg. Linksideal (resp. Rechtsideal), wenn es eine orthogonale Projektion $e\in \M$ gibt mit $\mathfrak{m}=\M e$ (resp. $\mathfrak{m}=e\M$) und dann ist $\mathfrak{m}$ genau dann zweiseitiges Ideal, wenn $e\in \mathrm{Z}(\M)$ (dann heißt $e$ \emph{zentrale Projektion}). Wir wissen dann außerdem bereits, dass $\mathfrak{m}$ als $\mathrm{swop}$-abgeschlossenes zweiseitiges Ideal sogar *-Ideal ist.

Es folgt: $\M$ Faktor gdw. $\M$ keine nicht-trivialen $\mathrm{swop}$-abgeschlossene 2-seitige Ideale enthält. Daraus folgt, dass jeder normale Homomorphismus auf $\M$ treu ist, also $\M$ einfaches Objekt in der Kategorie der vNAen!

Gilt $\mathrm{Z}(\M)\cong \ell^\infty(\{1,\dots,n\})$ (als vNAen!), so ist $\M$ direkte Summe von $n$ Faktoren ($\ell^\infty(\{1,\dots,n\})$ hat $n$ Extremalpunkte in der positiven Einheitskugel, diese werden zu $n$ orthogonalen Projektionen in $\mathrm{Z}(\M)$).

Ein Vektor $\xi\in \Hc$ heißt \emph{separierend}, falls aus $x\xi=0$ bereits $x=0$ für alle $x\in \M$ folgt. Es ist $\xi$ separierend für $\M$ genau dann, wenn $\xi$ zyklisch für $\M'$ ist. Beispiel: 1) $\mathrm{M}_n$ hat keinen zyklischen Vektor für $n\geq 2$, denn $\C \cdot \xi \neq \C^n$, also nicht zyklisch für die Kommutante. 2) $\xi = \sum_{i=1}^n e_i\otimes e_i = e_1 \oplus \dots \oplus e_n$ ist zyklisch und separierend für $\mathrm{M}_n \otimes 1 \subset \B(\C^n\otimes \C^n) = \B(\oplus_{i=1}^n \C^n)$, also auch separierend und zyklisch für seine Kommutante $1\otimes \mathrm{M}_n$.

\subsection{Gruppen-vNAen}

Sei $G$ Gruppe, dann definiert $G\ni g\mapsto L_g \coloneqq (\ell^2(G)\ni f\mapsto \delta_g\ast f)$ eine unitäre Gruppendarstellung auf $\ell^2(G)$. Dabei ist $L_g$ die \emph{linksreguläre Darstellung}. Definiere die Algebra $\mathrm{A}(G)\coloneqq \LH \{ L_g: g\in G\}$. Der Vektor $\delta_e$ ist zyklisch für $\mathrm{A}(G)$, außerdem gibt es noch eine rechtsreguläre Darstellung $R_g$ mit $\{R_g: g\in G\}\subset \mathrm{A}(G)'$, also $\delta_e$ zyklisch für $\mathrm{A}(G)'$, somit $\delta_e$ zyklisch und separierend für $\mathrm{A}(G)$, insbesondere $x\delta_e\mapsto x$ wohldefiniert: Schreibt man $x\in \mathrm{A}(G)$ als Matrix bzgl.\ der Basis $\delta_g$, so sind hängt der Eintrag $x_{g,h}$ nur von $gh^{-1}$ ab, also ist die Matrix \enquote{auf Diagonalen} konstant und eine Spalte (z.B. $\xi\coloneqq x\delta_e$) legt die Matrix bereits fest!

Es ist $\mathrm{M}(G)=\vN(\mathrm{A}(G))$ die \emph{von $G$ erzeugte vNA}. Es gilt $x\in \mathrm{Z}(\mathrm{M}(G))$ gdw. $\xi$ konstant auf den Konjugationsklassen von $G$ ist ($\xi$ ist Spalte der Darstellungsmatrix von $x$, also indiziert durch $G$ und alle Einträge des Vektors, deren Index in der gleichen Konjugationsklasse von $G$ liegt, sollen gleich sein). Eine Gruppe $G$ heißt \emph{\enquote{infinite conjugacy class}-Gruppe (ICC)}, falls die Konjugationsklassen von $e\neq g\in G$ unendlich sind. Es ist $\mathrm{M}(G)$ Faktor gdw. $G$ ICC: Ist $G$ ICC und $x\in \mathrm{M}(G)$ konstant auf Konjugationsklassen, so muss $\xi(g)=0$ für $g\neq e$ gelten, sonst $\xi \not\in \ell^2(G)$. Ist $G$ nicht ICC und $g$ habe endliche Konjugationsklasse $\mathrm{C}(g)$, so definiere $\xi$ mit $\xi(h)=1$ für $h\in \mathrm{C}(g)$ und $0$ sonst, dann $x\not \in \C 1_{\ell^2(G)}$, aber $x\in \mathrm{Z}(\mathrm{M}(G))$.

Ist $G=\Z$, dann ist $L_1=S$ der Linksshift. Es gilt: $$\mathrm{M}(\Z)=\overline{\mathrm{C}^*(1_{\ell^2(\Z)},S)}^\mathrm{stop}\cong \overline{\mathrm{C}(\mathbb{T})}^\mathrm{stop}=\mathrm{L}^\infty(\mathbb{T}).$$

Die Abbildung $\tau: \mathrm{M}(G)\ni x\mapsto \langle x\delta_e, \delta_e\rangle$ ist treue, normale Spur auf $\mathrm{M}(G)$.

\subsubsection{Beispiel einer ICC-Gruppe (ÜA 67)}

Sei $\mathrm{S}_\infty$ die Gruppe der endlichen Permutationen auf $\N$. Dann ist $\mathrm{S}_\infty$ ICC: Sei $\id\neq \sigma \in \mathrm{S}_\infty$, außerhalb $\{1,\dots,n\}$ identisch und $1\leq k\leq n$ mit $\sigma(k)\neq k$. Definiere $\rho\in \mathrm{S}_\infty$ mit $\rho(n+1)=k$, $\rho(k)=n+1$, sonst identisch. Dann $\rho^{-1}\sigma\rho\neq \sigma$ und gleiche Konstruktion mit $n+i$ für $i>1$ liefert Unendlichkeit der Konjugationsklasse!

\subsection{unendliches TP von $\mathrm{M}_k$}

TODO

\subsection{Spuren}

Ein positives lineares Funktional $\tau$ mit $\tau(xy)=\tau(yx)$ auf einer vNA $\M$ heißt \emph{Spur}. Sie heißt \emph{treu}, falls aus $\tau(x^*x)=0$ bereits $x=0$ folgt. Beispiel: Auf $\M=\mathrm{M}_n$ ist $\tau=\frac{1}{n} \Spur$ normalisierte Spur (also Zustand).

Eine Algebra mit treuer Spur besitzt keine \enquote{echten} Isometrien (wie der Rechtsshift auf $\ell^2(\N)$), d.h.\ jede Isometrie ist unitär: $u^*u=1$ und $1-uu^* \geq 0$, aus $\tau(1-uu^*)=\tau(1)-\tau(u^*u)=0$ folgt $uu^*=1$.

Auf Faktoren ist eine normale Spur automatisch treu: Nutze Aussage über zentralen Träger und dass in Faktoren $z_e=1_\M$ gilt (sonst wäre $\M z_e$ echtes $\sigma-\mathrm{swop}$-abgeschlossenes zweiseitiges Ideal).

\subsection{Vergleich von Projektionen}

Sind $p,q\in \M$ (orthogonale) Projektionen und gibt es partielle Isometrie $v\in \M$ mit $p=v^*v$ sowie $q=vv^*$, so sind $p$ und $q$ äquivalent, schreibe $p\sim q$. 

In $\B(\Hc)$: $p$ und $q$ sind genau dann äquivalent, wenn die Räume $p\Hc$ und $q\Hc$ die gleiche Dimension haben, denn $v$ ist ein isometrischer Isomorphismus zwischen den Räumen und umgekehrt gibt es zwischen Räumen gleicher Dimension einen unitären Isomorphismus. Jedoch gibt es mehr äquivalente Projektionen als unitär äquivalente Projektionen: Betrachte auf $\ell^2(\N)$ die Projektionen $1_{\ell^2(\N)}$ und $\mathrm{s}_l(S)$ wobei $S$ der Rechtsshift, dann ist $S$ partielle Isometrie zwischen den Räumen, aber es gibt keine unitäre Abbildung, die auf den Räumen Isometrie ist.

Für $x\in \M$ gilt $\mathrm{s}_l(x)\sim \mathrm{s}_r(x)$ ($x=v|x|$ und $\mathrm{s}_r(x)=\mathrm{s}_r(|x|)$). Sind $(p_i),(q_i)$ paarweise orthogonal und $p_i\sim q_i$, so gilt $\sum_i p_i \sim \sum_i q_i$ (nutze $\mathrm{stop}$-Limes beschränkter monotoner Netze). Damit lässt sich zeigen, dass $\preccurlyeq$ Ordnungsrelation ist (zeige Antisymmetrie mit Cantor-Bernstein-Schröder-von Neumann). Ist $\M$ Faktor (also einzige zentrale Projektion ist $1_\Hc$) und $0\neq p,q\in \M$, dann existiert $x\in \M$ mit $pxq\neq 0$ (d.h.\ es gibt keine \enquote{disjunkten} Projektionen in Faktoren!) und es gibt Projektionen $0\neq p_1,q_1\in \M$ mit $p_1\leq p$, $q_1\leq q$ und $p_1\sim q_1$ (setze $p_1\coloneqq \mathrm{s}_l(pxq)$ und $q_1\coloneqq \mathrm{s}_r(pxq)$).

Damit lässt sich das Vergleichstheorem zeigen: Es gilt immer $p\preccurlyeq q$ oder $q\preccurlyeq p$, d.h.\ die Ordnung ist total. Nutze dazu Zorn: Es gibt maximale Familie paarweise orthogonaler, äquivalenter Projektionen, die $\mathrm{stop}$-Limiten über die Summen können nicht beide ungleich $p$ und $q$ sein, sonst waren die Familien nicht maximal.

\subsubsection{Typen von Projektionen}

Eine Projektion $p\in \M$ heißt

\begin{itemize}
    \item \emph{endlich}, falls für $q$ mit $q\leq p, q\sim p$ folgt $p=q$. Analogie aus der Mengenlehre: $p$ ist nicht gleichmächtig zu einer echten Teilmenge!
    \item \emph{unendlich}, falls $p$ nicht endlich.
    \item \emph{echt unendlich}, falls $pz$ unendlich für alle zentralen Projektionen $z$. Eine unendliche Projektion in einem Faktor ist immer echt unendlich. Ist $\M$ unendlicher Faktor und $\Nc$ endlicher Faktor, so ist $1_\M \oplus 1_\Nc$ keine echt unendliche Projektion in $\M\oplus \Nc$, denn $0_\M \oplus 1_\Nc \in \mathrm{Z}(\M\oplus \Nc)$ und $0_\M\oplus 1_\Nc=1_\M \oplus 1_\Nc \cdot 0_\M \oplus 1_\Nc$ ist endlich.
    \item \emph{rein unendlich}, falls $p\neq 0$ und für $q$ endlich mit $q\leq p$ gilt $q=0$. Beispiel: $\C\cdot 1_\Hc$. In der Mengenlehre existiert sowas nicht: Wäre unendliche Menge, die keine endlichen Teilmengen besitzt!
    \item \emph{abelsch}, falls $p\M p$ kommutative Algebra. Beispiele: 1) $p=\chi_A$ in $L^\infty$ 2) $p$ ist Projektion auf eindimensionalen Teilraum.
    \item \emph{minimal}, falls aus $0\neq q \leq p$ bereits $p=q$ folgt. Entspricht Einpunktmengen in der Mengenlehre.
\end{itemize}

Ist $p$ abelsch, so auch endlich ($q\leq p$, $q\sim p$, $v:q\to p$, $v,v^*\in p\M p$, also $p=vv^*=v^*v=q$). Ist $p$ minimal, so ist $\dim p\M p=1$ (denn $p\M p$ wird von den Projektionen unterhalb $p$ erzeugt), insbesondere $p$ abelsch. Ist $\M$ Faktor, so folgt aus abelsch auch minimal ($\M$ Faktor, also auch $p\M p$ Faktor und kommutativ, also $\mathrm{Z}(p\M p)=p\M p$ und somit $\dim(p\M p)=1$ und damit $p$ minimal). Im allgemeinen stimmt die Umkehrung offensichtlich nicht, dann in einer kommutativen Algebra sind alle Projektionen abelsch und es gibt Projektionen mit Unterprojektionen. Ist $p$ endlich und $q\preccurlyeq p$, dann ist auch $q$ endlich.

\subsubsection{Typen von vNAen resp.\ Faktoren}

Eine vNA $\M$ heißt \emph{diskret/Typ I}, falls unter jeder zentralen Projektion eine abelsche Projektion liegt. Ist $\M$ Faktor, so ist dies äquivalent dazu, dass es eine minimale Projektion gibt (nur $1_\Hc$ ist zentrale Projektion und abelsche Projektion unterhalb ist auch minimal, wenn $\M$ Faktor). Andernfalls heißt $\M$ \emph{kontinuierlich}.

Ist $\M$ Faktor, so ergibt sich folgende Klassifikation:

\begin{table}[h]
    \centering
    \begin{tabular}{l|c||r}
        diskret & endlich & Typ $\Ifin$ \\
        diskret & unendlich & Typ $\Iinf$ \\
        \hline
        kontinuierlich & endlich & Typ $\IIfin$ \\
        kontinuierlich & unendlich, nicht rein unendlich & Typ $\IIinf$ \\
        \hline
        kontinuierlich & rein unendlich & Typ $\III$
    \end{tabular}
\end{table}

Beispiele: 1) $\mathrm{M}_n$ ist Typ $\Ifin$ 2) $\mathrm{L}^\infty(\Omega, \mu)$ ist Typ $\Ifin$ (da abelsche Projektionen immer endlich sind) 3) $\B(\Hc)$ mit $\dim \Hc = \infty$ ist Typ $\Iinf$ 4) $\mathrm{M}(S_\infty)$ ist Typ $\IIfin$: Wir haben treue normale Spur, also sind Isometrien unitär und somit erhalten wir Endlichkeit. Da Typ $\Ifin$ isomorph zu $\mathrm{M}_n$ und jene kompakte Operatoren besitzt, muss es $\IIfin$ sein 5) $\dim \Hc = \infty$ und $\M$ Typ $\IIfin$, dann $\M \otimes \B(\Hc)$ Typ $\IIinf$ 6) mit geeignetem treuen Zustand ist unendliches Tensorprodukt von $\mathrm{M}_k$ Typ $\III$.

An Beispiel 2 sieht man, dass unterhalb endlicher Projektionen noch sehr viele Projektionen liegen können, z.B. ist $\chi_{[0,1]}$ endliche Projektionen und durch $\chi_{[0,1/n]}$ sind unendlich viele verschiedene Projektionen gegeben, die alle unterhalb liegen!

\subsubsection{Klassifikation von diskreten Faktoren}

Zu einem diskreten Faktor $\M$ gibt es eine Indexmenge $J$ mit $\M\cong \B(\ell^2(J))$: Da der Faktor diskret ist, gibt es eine minimale Projektion $p\in \M$. Wir finden eine maximale Familie paarweise orthogonaler, zu $p$ äquivalenter, Projektionen. Die partiellen Isometrien $v_j$ erzeugen via $e_{ij}\coloneqq v_iv^*_j$ ein System von Matrixeinheiten, welches wiederum den Isomorphismus induziert. Die Indexmenge dieser Familie ist $J$. Da minimale Projektionen äquivalent sind, ist $J$ unabhängig von der Wahl der minimalen Projektion und somit ist $|J|$ vollständige Invariante. 

Ist $\A$ endlich-dimensionale C*-Algebra, so gilt $\A=\mathrm{M}_{n_1}\oplus \dots\oplus \mathrm{M}_{n_k}$: Durch Darstellung wird $\A$ zu Operatoralgebra, da endlich-dimensional auch vNA, das Zentrum ist endlich-dimensional und kommutativ, also $\mathrm{C}(\{x_1,\dots,x_k\})$, jeder Punkt entspricht einer zentralen Projektion, diese liefert Zerlegung in Typ $\Ifin$ Faktoren.

\subsubsection{Klassifikation kontinuierlicher Faktoren}

In einem diskreten Faktor gibt es eine minimale Projektion $p$. Zu dieser kann es keine Projektionen $p_1,p_2$ geben mit $p_1\sim p_2$ und $p_1+p_2=p$, denn aus $p$ minimal und $p_i\leq p$ folgt $p_i=0$. In kontinuierlichen Faktoren ist dies für jede Projektion möglich (Halbierungslemma) und ist die Projektion unendlich, so gibt es sogar $(p_n)_{n\in \N}$ mit $p_n\sim p$ und $p=\sum_n p_n$! Ist $(p_j)$ maximale Familie äquivalenter Projektionen, so gibt es Familie $(q_j)$ mit $p_j\sim q_j$ und $\sum_j q_j=1_\Hc$, also $\sum_j p_j \sim 1_\Hc$. Es folgt, dass alle unendlichen Projektionen äquivalent sind (zeige Äquivalenz zu $1_\Hc$), insbesondere gilt für einen Typ $\III$ Faktor $\M$ dann: $\mathcal{P}(\M)_{/\sim}=\{0,1\}$.

Ist $\M$ Faktor vom Typ $\IIfin$ und $\Kc$ Hilbertraum mit $\dim(\Kc)=\infty$, so ist $\Nc\coloneqq \M \otimes \B(\Kc)$ Faktor vom Typ $\IIinf$: $\Nc$ ist Faktor, da $\M$ und $\B(\Kc)$ Faktoren, nicht endlich (Shifte die Diagonale, wenn man $1_{\Hc\otimes \Kc}$ als Operatormatrix betrachtet), nicht reinunendlich (betrachte Operatormatrix mit endlich vielen $1_\Hc$ auf der Diagonale) und kontinuierlich aus Widerspruch mit Vergleichstheorem.

Umgekehrt ist jeder abzählbar zerlegbare Faktor vom Typ $\IIfin$ darstellbar als $\M\otimes \B(\ell^2(\N))$, es reicht also $\IIfin$ Faktoren zu verstehen.

Es ist $\M$ genau dann Typ endlicher Faktor, wenn $\M$ eine (treue) normale Spur besitzt und Spur ist vollständige Invariante für $\mathcal{P}(\M)$, also $p\preccurlyeq q$ gdw.\ $\tr(p)\leq \tr(q)$.

Für $\M$ Typ $\IIfin$ Faktor gilt $\tr(\mathcal{P}(\M))=[0,1]$ und $\tr$ ist eindeutig bestimmt: Durch sukzessive Halbierung von $1_\Hc$ erhalten wir Darstellung $1_\Hc=\sum_n q_n$ mit $\tr(q_n)=2^{-n}$. Durch dyadische Entwicklung von $\alpha\in [0,1]$ erhalten wir eine Projektion $p$ mit $\tr(p)=\alpha$ für jede normale Spur. Nun folgt für eine Projektion $q$ mit $\alpha\coloneqq \tr_1(q)$, dass $q\sim p_\alpha$, also $\tr_1(q)=\tr_1(p_\alpha)=\tr_2(p_\alpha)=\tr_2(q)$, also Eindeutigkeit der treuen, normalen Spur.

\begin{table}[h]
    \centering
    \begin{tabular}{cc}
        $\mathrm{I}_n$ & $\{0,\dots,n\}$ \\
        $\mathrm{I}_\infty$ & $\N \cup \{\infty\}$ \\
        $\IIfin$ & $[0,1]$ \\
        $\IIinf$ & $[0,\infty]$ \\
        $\III$ & $\{0,1\}$
    \end{tabular}
\end{table}


\end{document}
