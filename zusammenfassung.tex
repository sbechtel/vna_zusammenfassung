\documentclass[11pt,a4paper]{scrartcl}

\usepackage[utf8]{inputenc}
\usepackage[T1]{fontenc}
\usepackage[ngerman]{babel}
\usepackage{amsmath,amsthm,amssymb,dsfont}
\usepackage{mathtools}
\usepackage[paper=a4paper,left=25mm,right=25mm,top=25mm,bottom=25mm]{geometry}
\usepackage{float}
\usepackage{hyperref}
\usepackage{enumerate}
\usepackage{url}
\usepackage{tikz}
\usepackage{esint}
\usepackage{csquotes}
\usepackage{textcomp}

\usepackage{setspace}

\parindent 0pt
\linespread{1.5}

% Makros

\newcommand{\N}{\mathbb{N}} % natuerliche Zahlen
\newcommand{\Z}{\mathbb{Z}} % ganze Zahlen
\newcommand{\Q}{\mathbb{Q}} % rationale Zahlen
\newcommand{\R}{\mathbb{R}} % reelle Zahlen
\newcommand{\K}{\mathbb{K}} % Körper
\newcommand{\C}{\mathbb{C}} % komplexe Zahlen
\newcommand{\D}{\mathcal{D}}
\newcommand{\E}{\mathcal{E}}
\newcommand{\Hc}{\mathcal{H}}
\newcommand{\Kc}{\mathcal{K}}
\newcommand{\Sc}{\mathcal{S}}
\newcommand{\A}{\mathcal{A}}
\newcommand{\B}{\mathcal{B}}
\newcommand{\M}{\mathcal{M}}
\newcommand{\Nc}{\mathcal{N}}
\newcommand{\Rc}{\mathcal{R}}
\newcommand{\F}{\mathcal{F}}
\newcommand{\norm}[1]{\|#1\|}
\newcommand{\laplace}{\triangle}
\newcommand{\circum}{\text{\textasciicircum}}

% Umgebungen für Definitionen, Sätze, usw.

\theoremstyle{plain}
\newtheorem{thm}{Satz}[section]
\newtheorem*{lem}{Lemma}
\newtheorem{cor}[thm]{Korollar}
\newtheorem{prop}[thm]{Proposition}
\newtheorem*{ex}{Beispiel}
\newtheorem*{ntion}{Notation}

\theoremstyle{definition}
\newtheorem{defn}[thm]{Definition}

\theoremstyle{remark}
\newtheorem*{bem}{Bemerkung}
\newtheorem{bemnumber}[thm]{Bemerkung}

\def\Satzrefname{Satz}

\DeclareMathOperator{\supp}{supp}
\DeclareMathOperator{\esssupp}{ess supp}
\DeclareMathOperator{\id}{id}
\DeclareMathOperator{\loc}{loc}
\DeclareMathOperator{\pv}{pv}
\DeclareMathOperator{\grad}{grad}
\DeclareMathOperator{\Bil}{Bil}
\DeclareMathOperator{\Lin}{Lin}
\DeclareMathOperator{\Mult}{Mult}
\DeclareMathOperator{\LH}{LH}
\DeclareMathOperator{\Rang}{Rang}
\DeclareMathOperator{\Spur}{Spur}
\DeclareMathOperator{\vN}{vN}

\begin{document}

\title{Zusammenfassung von Neumann Algebren}
\author{Sebastian Bechtel}
\maketitle

\section{Erster Kontakt}

Eine *-Algebra $\M \subset \B(\Hc)$ heißt \emph{Operatoralgebra}. Ist $\M$ stop-abgeschlossen, so heißt $\M$ \emph{von Neumann Algebra (vNA)}. Durch GNS-Darstellung lässt sich jede C*-Algebra als Operatoralgebra darstellen. Sei $\Sc \subset \B(\Hc)$, da der Schnitt von vNAen wieder vNA ist, existiert kleinste vNA $\vN(\Sc)$, die $\Sc$ enthält. Ist $\Sc$ Operatoralgebra, so gilt $\vN(\Sc)=\overline{\Sc}^{stop}$, dies ist aber nicht offensichtlich, da Involution nicht stop-stetig ist!

Für $\M \subset \B(\Hc)$ ist die \emph{Kommutante} gegeben durch $\M'\coloneqq \{ x\in\B(\Hc): xy=yx \text{ für all } y\in \M \}$. Es gilt immer $\M'$ stop-abgeschlossen und ist $\M$ Operatoralgebra, so auch $\M'$, somit $\M'$ vNA. Ferner gilt $1_\Hc\in \M'$ und $\M'=\M'''$ (wegen $\M\subset \M''$), also gilt für $\M$ Operatoralgebra auch nach Bikommutantensatz (vgl. später) $\M'$ vNA! Aus $\M \subset \M''$ folgt für $\M$ Operatoralgebra, dass $\vN(\M)\subset \M''$. Ist $1_\Hc\in M$, so folgt mit Bikommutantensatz $\M''\subset \vN(\M)''=\vN(\M)$, also $\M''=\vN(\M)$. Ist $\M$ nicht s.a., geht alles schief: Die Matrizen $\left(\begin{smallmatrix} 0 & a \\ 0 & 0 \end{smallmatrix}\right)$ haben als Kommutante $\left(\begin{smallmatrix} b & c \\ 0 & b \end{smallmatrix}\right)$, also ist jene nicht s.a., somit folgt wegen $\M'=\M'''$, dass Bikommutantenbildung keine Selbstadjungiertheit herbeiführt.

\subsection{Beispiel: $L^\infty$ ist vNA}

Durch $L^\infty \ni g \mapsto M_g \in \B(L^2)$ wird $L^\infty$ als Operatoralgebra dargestellt, vgl. Borel-FK. Ist $(\Omega,\Sigma,\mu)$ lokalisierbar, so zerlege $L^2(\Omega)$ in direkte Summe $\oplus_i L^2(\Omega_i)$ mit $\mu(\Omega_i) < \infty$. Zeige dann für einen solchen Summanden, dass er seine eigene Kommutante ist, somit vNA: Sei $T\in L^\infty(\Omega)'$ und setze $f\coloneqq T(1)\in L^2(\Omega)$. Es gilt für $g\in L^\infty$ nun $T(g)=TM_g(1)=M_gT(1)=gf=M_f(g)$, also $T|_{L^\infty}=M_f|_{L^\infty}$, somit $M_f: L^\infty \subset L^2 \to L^2$ beschränkt. Wäre $f\not\in L^\infty$, würde es für $n\in \N$ messbare Mengen $\Omega_n \subset \Omega$ geben mit $|f(\omega)| \geq n$ f.ü. auf $\Omega_n$ und $g_n\coloneqq \chi_{\Omega_n}/\mu(\Omega_n)^{1/2}$ würde Beschränktheit auf $L^\infty$ widersprechen, also $f\in L^\infty$ und $M_f=T$ auf dichter Teilmenge, also $T\in L^\infty$.

\section{Tensorprodukte}

\subsection{algebraische Theorie}

Seien $E,F$ Vektorräume, $E^*,F^*$ ihre algebraischen Duale. Bezeichne mit $\Bil(E^*,F^*)$ die bilinearen Funktionale auf $E^*\times F^*$. Für $e\in E, f\in F$ definiere $e\otimes f\in \Bil(E^*, F^*)$ via $e\otimes f(e',f')\coloneqq e'(e)f'(f)$. Es heißt $e\otimes f$ \emph{elementarer Tensor} und $E\otimes F\coloneqq \LH \{e\otimes f: e\in E, f\in F\}$ heißt das \emph{Tensorprodukt} von $E$ mit $F$. Die Zuordnung $i: E\times F \ni (e,f) \mapsto e\otimes f \in E\otimes F$ ist bilinear und aus $0\neq e\in E, 0\neq f\in F$ folgt $e\otimes f \neq 0$, jedoch ist $i$ nicht injektiv. Ist $(e_i)$ Basis von $E$, $(f_j)$ Basis von $F$, dann $\{e_i\otimes f_j: i\in I, j\in J\}$ Basis von $E\otimes F$. Für einen weiteren Vektorraum $W$ gilt die Isomorphie $\Bil(E\times F, W)\cong \Lin(E\otimes F, W)$.

Darstellung eines Tensors ist nicht eindeutig, aber für eine minimale Darstellung (d.h. Anzahl der Summanden ist minimal) $x=\sum_{i=1}^n e_i\otimes f_i$ gilt $\{e_1,\dots,e_n\}$ sowie $\{f_1,\dots,f_n\}$ l.u. (und umgekehrt, vgl. endlich-Rang Operatoren!) und ist $0=\sum_{i=1}^n e_i\otimes f_i$ und $\{e_1,\dots,e_n\}$ l.u., so folgt $f_i=0$ für alle $i$.

\subsubsection{Tensorprodukt von linearen Abbildungen}

Für $A\in \Lin(E,E_1), B\in \Lin(F,F_1)$ gibt es eindeutige (universelle Eigenschaft!) lineare Abbildung $A\boxtimes B: E\otimes F \ni (e,f)\mapsto Ae\otimes Bf \in E_1\otimes F_1$. Wir wollen gerne $A\otimes B$ für $A\boxtimes B$ schreiben. Dazu identifizieren wir $\Lin(E,E_1)\otimes \Lin(F,F_1)$ mit einem Unterraum von $\Lin(E\otimes F, E_1\otimes F_1)$. Wegen $\Lin(E,E_1)\times \Lin(F,F_1) \ni (A,B) \mapsto A\boxtimes B \in \Lin(E\otimes F, E_1\otimes F_1)$ bilinear gibt es $\beta: \Lin(E,E_1)\otimes \Lin(F,F_1) \to \Lin(E\otimes F, E_1\otimes F_1)$ mit $\beta(A\otimes B) = A\boxtimes B$. Man zeigt, dass $\beta$ injektiv ist und kann dann wie gewünscht $A\otimes B$ mit $A\boxtimes B$ identizifieren.

\subsubsection{Tensorprodukt von Algebren}

Sind $\A,\B$ Algebren, so gibt es eindeutige Multiplikation auf $\A\otimes \B$ mit $x\otimes y \cdot x'\otimes y' = xx'\otimes yy'$.

\subsubsection{n-faches Tensorprodukt über Linearformen}

Erhalte Einbettung $E_1\otimes \dots \otimes E_n \hookrightarrow \Mult(E_1^*\times \dots \times E_n^*, \K)$ via $m(e_1\otimes \dots \otimes e_n) \coloneqq ((e_1', \dots, e_n')\mapsto e_1'(e_1)\cdot \dots \cdot e_n'(e_n)$. Dann $E_1\otimes \dots \otimes E_n \cong m(E_1\otimes \dots \otimes E_n)$.

\subsubsection{Tensorprodukt über endlich-Rang Operatoren}

Für $e\in E, f\in F$ definiere $t_{e,f}: E^*\ni e' \mapsto \langle e, e' \rangle f$, also $t_{e,f}\in \Lin(E^*,F)$. Es ist $E\times F \ni (e,f) \mapsto t_{e,f}$ bilinear, also gibt es $\beta: E\otimes F \to \Lin(E^*,F)$ mit $\beta(e\otimes f) = t_{e,f}$, welches injektiv ist, also $E\otimes F \hookrightarrow \Lin(E^*,F)$.

Ist $E\neq E^{**}$, so ist $\beta$ sicher nicht surjektiv!

Der \emph{Rang eines Tensors} $x\in E\otimes F$ ist gegeben durch $\Rang(\beta(x))$ und stimmt mit der Länge einer minimalen Darstellung überein.

Betrachte nun $\beta: E^*\otimes F \to \F(E^{**},F)$. Diese ist nach wie vor injektiv, aber i.A. nicht surjektiv. Wenn wir jedoch $\beta(e'\otimes f)|_{E\subset E^{**}}$ betrachten, so bleibt die Zuordnung injektiv und wird sogar surjektiv, also $E^*\otimes F \cong \F(E,F)$. Wir können also die endlich-Rang Operatoren als Tensorprodukt verstehen! Dies gilt ferner für die topologischen Dualräume.

\subsubsection{Beispiel: Matrizen als TP und ihre Spur}

Betrachte $(\K^n)^* \otimes \K^m$. Es ist $t_{e_j',e_i}=e_{ij}$, also $m\times n$ Matrizen sind Tensorprodukt. Ist $A$ eine $m\times n$-Matrix mit Zeilen $a_1,\dots,a_m \in (\K^n)^*$, so ist $A=\sum_{j=1}^m a_j \otimes e_j$, analog: sind $b_1,\dots,b_n \in \K^m$ die Spalten von $A$, so gilt $A=\sum_{i=1}^n e_i \otimes b_i$. Definiere $(\K^n)^* \times \K^n \ni (x',y)\mapsto \langle y, x' \rangle$ bilinear, diese besitzt Fortsetzung $\tau: (\K^n)^*\otimes \K^n = M_n \to \K$ mit $\tau(A)=\Spur(A)$. Nutze dies später, um den Begriff der Spur zu verallgemeinern!

\subsection{topologische Tensorprodukte}

\subsubsection{Tensorprodukte von Hilberträumen}

Bezeichne mit $\odot$ das algebraische Tensorprodukt. Sind $\Hc,\Kc$ Hilberträume, so gibt es auf $\Hc\odot \Kc$ ein eindeutiges Skalarprodukt $\langle \cdot, \cdot \rangle_{HS}$ mit $\langle x\otimes y, x'\otimes y' \rangle_{HS} = \langle x,x' \rangle_\Hc \langle y, y' \rangle_\Kc$. Also ist $(\Hc\otimes \Kc, \langle \cdot, \cdot \rangle_{HS})$ Prähilbertraum. Dessen Vervollständigung heißt \emph{Hilbertraumtensorprodukt} von $\Hc$ und $\Kc$, schreibe $\Hc \otimes_{HS} \Kc$ bzw. $\Hc \otimes \Kc$. Die induzierte Norm $\|\cdot\|_{HS}$ ist eine \emph{Kreuznorn}, d.h. $\|x\otimes y\|_{HS}=\|x\|_\Hc \|y\|_\Kc$ (somit ist $B: \Hc \times \Kc \to W$ bilinear genau dann beschränkt, wenn es $T_B: \Hc \odot \Kc \to W$ ist, also erhalten wir eine topologische Version der universellen Eigenschaft für $\Hc \otimes \Kc$). Außerdem folgt aus $x_i \to x, y_j\to y$, dass $x_i \otimes y_j \to x\otimes y$ und sind $(e_i), (f_j)$ ONBs von $\Hc$ bzw. $\Kc$, so ist $\{ e_i \otimes f_j: i\in I, j\in J \}$ ONB von $\Hc\otimes \Kc$.

Beispiele: 1) Es gilt $l^2(I)\otimes l^2(J) = l^2(I\times J)$ (Dimensionsvergleich!) 2) Sind $(\Omega_1, \Sigma_1,\mu_1), (\Omega_2,\Sigma_2,\mu_2)$ $\sigma$-endliche Maßräume, dann gilt $L^2(\Omega_1\times \Omega_2) \cong L^2(\Omega_1)\otimes L^2(\Omega_2)$ und der unitäre Operator ist eindeutig mit $U(f_1\otimes f_2) = f_1(\omega_1)f_2(\omega_2)$ (nutze topologische universelle Eigenschaft!)

Sind $A\in \B(\Hc), B\in \B(\Kc)$, denn gibt es eindeutigen Operator $A\otimes B\in \B(\Hc\otimes \Kc)$ mit $A\otimes B(x\otimes y)=Ax\otimes By$ und es gilt $\|A\otimes B\|=\|A\|\|B\|$. Eindeutigkeit folgt aus der Eindeutigkeit auf dem dichten algebraischen Tensorprodukt. Zeige $\|A\otimes B\|=\|A\|\|B\|$ auf $\Hc \odot \Kc$, dann stetige Fortsetzung. Nutze $A\otimes B=(A\otimes 1_\Kc)(1_\Hc\otimes B)$ (denn $((A\otimes 1_\Kc)(1_\Hc\otimes B))(x\otimes y)=A\otimes 1_\Kc(x\otimes By)=Ax\otimes By$) und $\|A\otimes 1_\Kc\| \leq \|A\|$. Außerdem gilt $(A\otimes B)^*=A^*\otimes B^*$.

\subsection{Tensorprodukte auf Banachräumen}

Seien $E,F$ Banachräume. Auf $E\odot F$ wird durch $\|z\|_\pi \coloneqq \inf \{ \sum_i \|x_i\|_E \|y_i\|_F: z = \sum_i x_i \otimes y_i \}$ eine Kreuznorm definiert, welche \emph{projektive Norm} oder \emph{maximale Norm} heißt. Motivation: Für eine Kreuznorm $\|\cdot\|$ gilt $\|z\|\leq \sum_i \|x_i\|\|y_i\|$ wegen Dreiecksungleichung und Kreuznormeigenschaft. Beachte, dass Supremum keinen Sinn machen würde, da wegen $0=x\otimes y + (-x)\otimes y$ für $x,y\neq 0$ bereits die Definitheit verletzt wäre.

Wiederum auf $E\odot F$ definieren wir eine weiter Norm, genannt \emph{injektive Norm} oder \emph{$\varepsilon$-Tensornorm}, durch $\|z\|_\varepsilon \coloneqq \sup \{ |e'\otimes f'(z)|: e'\in E^*_1, f'\in F^*_1 \}$. Die Kreuznormeigenschaft folgt aus Hahn-Banach: zu $x\in E, y\in F$ gibt es $e'\in E^*_1, f'\in F^*_1$ mit $e'(x)=\|x\|_E, f'(y)=\|y\|_F$, also $\|x\otimes y\|_\varepsilon \geq |e'\otimes f'(x\otimes y)|=\|x\|_E \|y\|_F$.

\section{Die vNA $\B(\Hc)$}

\subsection{Projektionen}

Ein $p\in \B(\Hc)$ heißt \emph{Projektion}, falls $p^2=p$ gilt und \emph{orthogonale Projektion}, falls $p$ Projektion und $p^*=p$. Man nennt $v\in \B(\Hc)$ eine \emph{partielle Isometrie}, falls $v^*v$ eine orthogonale Projektion ist (automatisch s.a., Projektion ist zu prüfen). Ist $v$ Isometrie so ist $\Nc(v)^\perp$ der Anfangsraum von $v$ und $\overline{\Rc(v)}$ der Zielraum von $v$. Es ist $v^*v$ orthogonale Projektion auf den Anfangsraum, genannt \emph{initiale Projektion}, und $vv^*$ ist orthogonale Projektion auf den Zielraum, genannt \emph{finale Projektion}. Zwischen Anfangsraum und Zielraum ist eine partielle Isometrie eine Isometrie.

Für $x\in \B(\Hc)$ definieren wir Anfangsraum und Zielraum wie oben und bezeichnen die initiale Projektion als \emph{Rechtsträger} $s_r(x)$ und die finale Projektion als \emph{Linksträger} $s_l(x)$. Für $x$ s.a. definiere den Träger $s(x)\coloneqq s_l(x)=s_r(x)$ (wegen $\Nc(x)^\perp=\overline{\Rc(x^*)}=\overline{\Rc(x)}$). Es gilt (analog zu den partiellen Isometrien!) $s_l(x)=s(xx^*)$ und $s_r(x)=s(x^*x)$.

\subsubsection{polare Zerlegung}

Eine komplexe Zahl $z$ lässt sich schreiben als $z=e^{i\varphi}|z|$, also $|z| \geq 0$ und

% TODO polare Zerlegung fertig schreiben. Motivation/Analogie zu C: MSE-Thread


\end{document}
